\section{Исходный код}

Программа реализует сортировку подсчетом для пар \enquote{ключ-значение}, где ключ — 32-битное беззнаковое целое число, а значение — 64-битное беззнаковое целое число.

Основная идея алгоритма: для каждого ключа подсчитывается количество элементов с таким ключом, затем вычисляются позиции элементов в отсортированном массиве путем накопления сумм счетчиков. Элементы размещаются в выходном массиве согласно вычисленным позициям.

Т.к код не помещается на одну страницу, приведена таблица с описанием функций и классов:

\begin{longtable}{|p{7.5cm}|p{7.5cm}|}
\hline
\rowcolor{lightgray}
\multicolumn{2}{|c|}{contest.hpp}\\
\hline
\texttt{template <typename T> class TVector} & Класс динамического массива с методами \texttt{PushBack}, \texttt{Reserve}, \texttt{Size}, \texttt{Empty}, операторами индексирования и итераторами\\
\hline
\texttt{TVector(size\_t count, const T \&value)} & Конструктор, создающий вектор из \texttt{count} элементов со значением \texttt{value}\\
\hline
\texttt{void Reserve(size\_t newCapacity)} & Резервирование памяти для \texttt{newCapacity} элементов\\
\hline
\texttt{void PushBack(const T \&value)} & Добавление элемента в конец вектора\\
\hline
\texttt{void CountingSort(TVector<TElement> \&arr)} & Функция сортировки подсчетом. Находит минимум и максимум ключей, создает массив счетчиков, подсчитывает элементы, вычисляет позиции и размещает элементы в отсортированном порядке\\
\hline
\texttt{bool CompareById(const TElement \&left, const TElement \&right)} & Функция сравнения элементов по полю \texttt{id}\\
\hline
\rowcolor{lightgray}
\multicolumn{2}{|c|}{contest\_main.cpp}\\
\hline
\texttt{int main()} & Считывает пары ключ-значение из стандартного ввода с помощью \texttt{scanf}, вызывает сортировку, выводит отсортированные данные с помощью \texttt{printf}\\
\hline
\end{longtable}

Структура для хранения элементов:
\begin{lstlisting}[language=C++]
struct TElement {
    uint32_t id;
    uint64_t value;
};
\end{lstlisting}

Объявление класса TVector (без реализации методов):
\begin{lstlisting}[language=C++]
template <typename T> 
class TVector {
  public:
    TVector() = default;
    TVector(size_t count, const T &value);
    TVector(const TVector &other);
    TVector(TVector &&other) noexcept;
    TVector &operator=(const TVector &other);
    TVector &operator=(TVector &&other) noexcept;
    ~TVector();
    
    void PushBack(const T &value);
    void PushBack(T &&value);
    void Reserve(size_t newCapacity);
    size_t Size() const;
    size_t Capacity() const;
    bool Empty() const;
    T &operator[](size_t index);
    const T &operator[](size_t index) const;
    T *Begin();
    const T *Begin() const;
    T *End();
    const T *End() const;
    void Swap(TVector &other) noexcept;
    
  private:
    std::allocator<T> allocator_;
    T *data_ = nullptr;
    size_t size_ = 0;
    size_t capacity_ = 0;
};
\end{lstlisting}

\pagebreak

\section{Консоль}
\begin{alltt}
$ make
g++ .obj/contest_main.o -o solution

$ ./solution < cases/5.txt
019383	3546839058697536693
205057	16563277856680048351
219581	8725518293060573749
526830	3189070866999033975
690718	11796781922971975819

$ cat cases/5.txt
690718	11796781922971975819
219581	8725518293060573749
205057	16563277856680048351
526830	3189070866999033975
019383	3546839058697536693
\end{alltt}
\pagebreak
